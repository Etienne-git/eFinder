\documentclass[12pt, a4paper, oneside]{article}
%die Pakete werden hier durch den Include-Befehl separat eingelesen
\usepackage[utf8]{inputenc}
\usepackage{amsmath} % Mathematik-Pakete
\usepackage{amsfonts}
%\usepackage{mathptmx} %times new roman
%\usepackage[T1]{fontenc} %vollen Zeichensatz ]
\usepackage{amssymb}
\usepackage[ngerman]{babel}
\usepackage{graphicx}
\usepackage{microtype} %besserer Randausgleich
\usepackage{footnote}
\usepackage{blindtext}
\usepackage{etoolbox}
%\usepackage{makeidx}
%\usepackage{dsfont}
\usepackage{lettrine}%\usepackage{geometry}
%\usepackage{xcolor} % Für verschiedene Farben
\newcommand{\ricardo}[1]{\colorbox{ForestGreen}{\color{white}   \textsf{\textbf{Ricardo}}} \textcolor{ForestGreen}{#1}}
\usepackage[pdfborderstyle={/S/U/W 1}]{hyperref} % Für interaktive Refernzierung im PDF
\usepackage{csquotes}
\usepackage{acro}
\usepackage{hyperref} % Für interaktive Refernzierung im PDF
\usepackage[onehalfspacing]{setspace}%Zeilenabstand 1.5
% \usepackage{picins} % Das Umfließen einer Grafik im Text kann mit dem Paket PicIns erreicht werden.
%\usepackage{fontspec} 

%\usepackage[utf8]{inputenc}
\usepackage[ngerman]{babel}
\usepackage[top=2.5cm, bottom=2.5cm, left=2cm, right=3.5cm]{geometry}
\usepackage{bibgerm}
\usepackage{tabularx}
\usepackage{adjustbox}
\usepackage{cite}
\usepackage{blindtext}
\usepackage{chngcntr}
\usepackage{epsfig}
\usepackage{longtable}
\usepackage{natbib}
%\usepackage{showframe}
\usepackage{dcolumn}%benötigt für stargaze
\usepackage{here}%lädt das Paket zum Erzwinge n der Grafikposition
\usepackage{floatflt}%Bilder im Fließtext
%\usepackage{fontspec}
%\usepackage{fontenc}
\usepackage{dsfont}
%\setsansfont[Ligatures=TeX]{Arial}
%\renewcommand{\familydefault}{\sfdefault}
%\usepackage{times}
\usepackage{graphicx}
\usepackage{epstopdf}

\usepackage{xcolor}
\usepackage{listings}

\usepackage{lipsum}




\lstset{language=R}
\lstset{basicstyle=\ttfamily,
basicstyle=\small}
\lstset{literate=%
  {Ö}{{\"O}}1
  {Ä}{{\"A}}1
  {Ü}{{\"U}}1
  {ß}{{\ss}}1
  {ü}{{\"u}}1
  {ä}{{\"a}}1
  {ö}{{\"o}}1
}

\title{\textbf{Titel der Arbeit}}
\author{Max Mustermann}

\setlength{\parindent}{0cm} %keine Einrückung
\linespread{1.5} 
\acsetup{first-style=short}
\newpage


\newpage
%in alphabetischer Reihenfolge

\newcounter{SeitenzahlSpeicher}
\begin{document}

 \thispagestyle{empty}
\begin{titlepage}
	 \thispagestyle{empty}
	% thispagestyle{empty} unterdrückt Seitenzahlen auf der gewünschten Seite
	\newfont{\smc}{cmcsc10 at 12pt}
%	\maketitle
	%Aufpassen mit fi: Fehlercode U+FB01
	%%%%%%%%%%%%%%%%%%%%%%%%%%%%%%%%%%%%%%%%%%
%Titel, Autor, Seminar, Semester, Dozent %
%%%%%%%%%%%%%%%%%%%%%%%%%%%%%%%%%%%%%%%%%%
\begin{center}
	 \thispagestyle{empty}
\begin{figure}[t]
	\centering
	\includegraphics[width=0.6\textwidth]{Hda_logo.svg.png}
	
\end{figure}

$~~$\\
\paragraph{}$~~$\\
\textbf{\huge Projektdokumentation eFinder}\paragraph{}$~~$\\
\paragraph{}$~~$\\
\paragraph{}$~~$\\
\textbf{im Rahmen des Fachs Nutzerzentrierte Softwareentwicklung}\\ \textbf{Fachbereich Informatik}\\ \textbf{Hochschule Darmstadt}
\paragraph{}$~~$\\
\paragraph{}$~~$\\
\paragraph{}$~~$\\
\paragraph{}$~~$\\
\text{von: Lukas Räpple \& Etienne Gotha}\\
\text{Dozent: Prof. Dr. Hans-Peter Wiedling}\\
\end{center}	
\end{titlepage}



\begin{spacing}{1}
\pagenumbering{Roman}
\setcounter{page}{2}
\tableofcontents
\end{spacing}
\newpage
\begin{spacing}{1}
\section*{Abbildungsverzeichnis} 
\addcontentsline{toc}{section}{Abbildungsverzeichnis}
\renewcommand{\listfigurename}{}
\listoffigures
\end{spacing}
\newpage
\clearpage
\setcounter{SeitenzahlSpeicher}{\value{page}}
\pagenumbering{arabic}
\newpage
\pagenumbering{arabic}

\section{Einführung}
Das vorliegende Projekt entstand im Rahmen des Praktikums zur Veranstaltung Nutzerzentrierte Softwareentwicklung im Wintersemester 2021/2022 an der Hochschule Darmstadt. Im Folgenden wird der Ablauf des Projektes von der Konzeption bis zur Auswertung der Ergebnisse dargestellt.

\subsection{Projektbeschreibung}
Gegenstand des Projektes ist die Entwicklung einer prototypischen Anwendung, die verschiedenen Nutzern den Zugang zu Informationen und Diensten bezüglich Ladestationen für Elektrofahrzeuge bereitstellt. Als Entwicklungsplattform wird Android Studio mit der Programmiersprache Java verwendet. Die Applikation soll sich dabei gleichermaßen an Fahrer von Elektrofahrzeugen als auch an Servicekräfte, die für den Betrieb der Ladestationen zuständig sind, richten. Ziel des Projektes ist es einen durchgehend auf diese Nutzergruppen ausgerichteten Entwicklungsprozess zu durchlaufen, um ein möglichst hohes Maß an Usability zu erreichen.

\subsection{Vorgehensweise}
Die Durchführung des Projektes und somit auch der vorliegenden Dokumentation orientiert sich am Wasserfallmodel. Das Model kann in die fünf Phasen Analyse, Design, Implementierung, Test und Betrieb unterteilt werden \cite[S. 16 ff.]{SoftwareDevelopmentBestPractices}. Da es sich beim vorliegenden Projekt um die Erstellung eines Prototypen handelt, wird allerdings auf eine Betrachtung der Phase Betrieb verzichtet.\\

Im ersten Schritt des Projektes wird eine heuristische Anforderungsanalyse durchgeführt. Dabei werden begründete Annahmen darüber getroffen, in welchem Kontext die Anwendung vermutlich verwendet wird und welche Intentionen die einzelnen Nutzergruppen mit sich bringen.\\

Die darauf folgende Designphase befasst sich zunächst mit den Fragen, wer der durchschnittliche Benutzer ist und welche Ziele er mit der Benutzung der Anwendung verfolgt. Dafür wird einerseits ein durchschnittliches Nutzerprofil in Form einer Persona für jeweils den Elektroautofahrer und den Servicemitarbeiter erstellt. Außerdem werden User Stories zu den beiden Nutzergruppen verfasst, die in Form von kurzen Statements Bedürfnisse darlegen, die der Nutzer mit der Anwendung befriedigen möchte. Auf Basis dieser Ergebnisse wird dann ein Paper Prototyp erstellt, der den Aufbau der einzelnen Ansichten/Screens skizziert.\\

Für die Implementierung des zuvor erstellten Designs werden im ersten Schritt Überlegungen zum Aufbau der Software und der Kopplung einzelner Funktionalitäten aufgestellt. Danach wird die Anwendung agil entwickelt, indem möglichst früh ein fertiges Produkt geliefert wird. Dieses Produkt wird dann im Rahmen des vorhandenen Zeithorizonts iterativ durch weitere Komponenten ergänzt. Zur Verwaltung des kooperativ erstellten Codes wird ein Git Repositorium verwendet, wo die einzelnen Sprints als Issues angelegt sind.\\

In der letzten Phase des Projektes wird die Anwendung hinsichtlich ihrer Funktionalität und Usability getestet und bewertet.\\

\section{Analyse}
\subsection{Heuristische Anforderungsanalyse}
Betrachten wir zunächst die Umstände, die den Fahrer eines Elektroautos dazu verleiten könnten eine App für Ladestationen zu nutzen. Eine Intention, die als häufiger Kontext für die Verwendung der App angenommen werden kann, ist die Suche nach einer Ladestation in der näheren Umgebung des Nutzer. Eine leere Batterie, eventuell noch in einer unbekannten Umgebung, sind wieder auftrettende Situationen, die eine schnelles Auffinden der nächsten Ladesäule nötig machen.\\

Am 01.01.2022 sind im Bundesgebiet 44.486 Ladesäulen gemeldet. Damit ist die Auswahl an Ladestationen in einem Jahr um 10.048 gestiegen. Allerdings ist der Ausbau der Infrastruktur im Vergleich der Bundesländer und vor allem im Vergleich zwischen Stadt und Land ungleich verteilt \cite[]{Ladeinfrastruktur2022}. Diese inkonsistente Netzabdeckung bringt Unsicherheitein für den Fahrer eines Elektroautos mit sich. Somit wird ein potentieller Nutzer von der App außerdem erwarten, dass er sich die Ladesäulendichte in einem weiter entfernten Reiseziel anzeigen lassen kann.\\

Desweiteren gibt es bei den einzelnen Ladesäulen deutliche Unterschiede. Einerseits hinsichtlich des Preises und andererseits hinsichtlich angebotener Steckertypen und Ladegeschwindigkeit. Aus der wachsenden Anzahl an Optionen eine informierte Entscheidung treffen zu können, ist somit ein weiterer Anreiz für die Verwendung der App.\\

\section{Design}
\subsection{User Story Mapping}
\subsection{Persona}
\subsection{Paper-Prototyping}

\section{Implementierung}
\subsection{Softwaredesign}
\subsubsection{Klassen \& Kopplung}
\subsection{Umsetzung der Screens und Funktionalitäten}
\subsubsection{Navigationsleiste}
\subsubsection{Registrierung \& Login}
\subsubsection{Datenbankintegration}
\subsubsection{Map-Screen}
\subsubsection{Such-Screen}
\subsubsection{Admin-Screen}
\subsubsection{Favoriten-Screen}
\subsubsection{Einstellungs-Screen}

\section{Test}
\subsection{Funktionale Tests}
\subsection{Usability-Test}
\newpage

\section{Fazit}

\section{Literaturverzeichnis}
\bibliographystyle{natdin}
\renewcommand{\refname}{} 
\bibliography{bibba}

\newpage


\appendix %Anhang
\pagenumbering{arabic}
\section{Anhang}

\newpage

\section{Eigenständigkeitserklärung}
``Ich versichere, dass ich die Arbeit selbständig und ohne Benutzung anderer als der angegebenen Hilfsmittel angefertigt habe. Alle Stellen, die wörtlich oder sinngemäß aus Veröffentlichungen oder anderen Quellen entnommen sind, sind als solche kenntlich gemacht. Die schriftliche und die elektronische Form der Arbeit stimmen überein. Ich stimme der Überprüfung der Arbeit durch eine Plagiatssoftware zu.''
\paragraph{}$~~$\\
\paragraph{}$~~$\\
\vspace{50pt} 
\noindent\rule{5cm}{.4pt}\hfill\rule{5cm}{.4pt}\par 
\noindent Ort, Datum \hfill Unterschrift 
\end{document}